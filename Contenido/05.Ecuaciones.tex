Siguiendo con el desmenuce de la sintaxis, vamos a hablar de ecuaciones,
el motivo por el que mucha gente (\emph{científicos}) se pasan a LaTeX.
Como dijimos el otro día, hay dos tipos de ecuaciones en LaTeX:

\begin{itemize}
\item
  las que van \textbf{dentro de una línea}, que se escriben entre signos
  de dólar y se suelen conocer como \emph{inline}
\item
  las que tienen \textbf{línea propia}, que usan el entorno
  \lstinline!equation! o el atajo\footnote{Yo no suelo usar el atajo
    porque me resulta más difícil de leer, pero, oyes, para gustos
    colores.} \lstinline!\[...\]!
\end{itemize}

Aquí tenemos un ejemplo usando los dos tipos:

\begin{lstlisting}[language={[latex]tex}]
\begin{equation}

  e^{i\pi} + 1 = 0

\end{equation}

donde $i =\sqrt{-1}$
\end{lstlisting}

Como veis, escribimos las ecuaciones mediante comandos, algo que
inicialmente parece un atraso pero que cuando cogemos un poco de
práctica, es terriblemente eficaz. Si estáis usando un editor
específico, tendréis una barra con los símbolos más usados, es una buena
forma de empezar con las ecuaciones. Más abajo os hablo de la sintaxis
más en detalle y doy unos ejemplos. Al escribir ecuaciones es
recomendable cargar los siguientes paquetes:

\begin{itemize}
\item
  \href{https://www.ctan.org/pkg/amsmath}{\lstinline!amsmath!} (AMS
  Math), que mejora el comportamiento y el aspecto de las ecuaciones.
  Nos permite, por ejemplo, añadir un asterisco en el entorno
  \lstinline!equation! para crear ecuaciones sin numerar.
\item
  \href{https://www.ctan.org/pkg/amsthm}{\lstinline!amsthm!} (AMS
  Theorem), que define los entornos teorema y demostración.
\item
  \lstinline!amssymb! (AMS Symbol), que carga a su vez
  \lstinline!amsfonts! e incluye una colección de símbolos matemáticos.
\end{itemize}

Podemos cargarlos todos a la vez añadiendo esta línea al
\emph{preámbulo}\footnote{Recordemos: el \emph{preámbulo} es lo que hay entre la
definición del documento (\lstinline!\\documentclass!) y el inicio del entorno
\lstinline!document! (\lstinline!\\begin\{document\}!).}:

\begin{lstlisting}[language={[latex]tex}]
\usepackage{amsmath, amsthm, amssymb}
\end{lstlisting}

Ese AMS que precede a todos ellos viene de
\href{http://www.ams.org/home/page}{\emph{American Mathematical
Society}}, los que originalmente desarrollaron estos paquetes.

\section{Comandos}\label{sec:comandos}

Vamos a ver un poco de sintaxis, pero antes de nada os dejo un par de
herramientas interesantes sobre todo para los novatillos (o
\emph{Nóbeles} que decía mi profe de autoescuela, \emph{conductor Nóbel}
(sic)):

\begin{itemize}
\item
  \textbf{Editores de ecuaciones online}: hasta que le vayamos cogiendo
  el callo a las ecuaciones, aparte de la barrita del IDE tenemos
  editores online como
  \href{http://www.numberempire.com/texequationeditor/equationeditor.php}{este}
  o
  \href{http://www.numberempire.com/texequationeditor/equationeditor.php}{este
  otro} que es más cuco.
\item
  \textbf{Detexify}: si no sabemos cómo se llama un símbolo y, por lo
  tanto, no podemos buscar su comando tenemos
  \href{http://detexify.kirelabs.org/classify.html}{Detexify}, un
  cacharro en el que pintamos el símbolo que estamos buscando y nos
  localiza los más parecidos. Especialmente útil con la típica duda de
  \emph{¿esa letra es xi o chi?} o mi favorita \emph{¿cómo se llama la R
  esa gorda de los números reales?}. Hacemos el dibujillo y hala.
\end{itemize}

\subsection{Símbolos comunes}\label{sec:simbolos}

Símbolos hay a pilas, os voy a poner unos pocos comandos aquí pero lo
mejor es que hurguéis.

\begin{itemize}
\item
  \emph{Sumas, restas y exponenciales}: se hacen con el símbolo de toda
  la vida \lstinline!+!, \lstinline!-! y \lstinline!^!
\item
  \emph{Multiplicaciones}: aquí hay variedad según los gustos, si
  queremos el punto usamos el comando \lstinline!\cdot! si nos gusta más
  el aspa usamos \lstinline!\times!. Hacedme un favor y no me uséis ni
  la equis ni el asterisco.
\item
  \emph{Raíces}: se hacen con el comando \lstinline!\sqrt{argumento}! si
  son raíces cuadradas y añadiendo el numerito como argumento opcional
  (es decir, entre corchetes) para cualquier otra
  \lstinline!\sqrt[raíz]{argumento}!
\item
  \emph{Integrales}: funcionan con el comando \lstinline!\int!, si
  queremos que tengan límites definidos no tenemos más que escribir
  \lstinline!\int_{inferior}^{superior}!. Por ejemplo, esta integral
  impropia \(\int_{0}^{\infty}\) se conseguiría así
  \lstinline!\int_{0}^{\infty}!. Si os fijáis las integrales, a
  diferencia de las raíces, no llevan llaves. Esto ocurre porque la raíz
  necesita saber cómo de largo es el contenido, la integral es
  simplemente el chirimbolo.
\item
  \emph{Sumatorios}: son como las integrales pero con el comando
  \lstinline!\sum!
\item
  \emph{Fraciones}: tan sencillas como
  \lstinline!\frac{numerador}{denominador}!
\end{itemize}

Tenéis en las referencias listas de símbolos para que les echéis una
ojeada si os parece.

\subsection{Letras griegas}\label{sec:letrasGriegas}

Una de las mejores cosas de LaTeX en mi opinión es su método para
escribir letras griegas, tan sencillo como escribir su nombre en
minúsculas para la letra en minúscula y ponerle la primera en mayúscula
para una letra griega en mayúscula. Se entenderá mejor con un ejemplo:

\begin{quote}
\lstinline!\omega! crea ω (\emph{omega minúscula}) y \lstinline!\Omega!
crea a su vez Ω (\emph{omega mayúscula})
\end{quote}

\subsection{Operadores}\label{sec:operadores}

Los operadores son las funciones cuyo nombre se escribe en texto, como
\emph{sin} o \emph{ln}. LaTeX tiene algunos de ellos definidos y es
importante usarlos para que las ecuaciones nos queden bien. Va un
ejemplo:

\begin{lstlisting}[language={[latex]tex}]
\sin^2 x + \cos^2 x = 1
\end{lstlisting}

Que crea: 

\begin{equation*}
\sin^2 x + \cos^2 x = 1 
\end{equation*}

\subsection{Matrices}\label{sec:matrices}

Funcionan de manera similar a las tablas (las columnas se separan con el
ampersand y se salta de línea con \lstinline!\\!), pero usan el entorno
\lstinline!matrix! y relacionados. El entorno \lstinline!matrix! nos
crea una matriz sin delimitadores, tendríamos que añadírselos nosotros.
Los entornos \lstinline!pmatrix!, \lstinline!vmatrix!,
\lstinline!Vmatrix! \lstinline!bmatrix! y \lstinline!Bmatrix! nos añaden
respectivamente paréntesis, barras\footnote{Como las de un determinante},
barras\footnote{Como las de una norma} dobles, corchetes y llaves. Estos
entornos que cito centran el contenido, si quisiéramos cambiar la
alineación tendríamos que usar su variantes con asterisco y darle un
argumento. Este sería un ejemplo de una matriz sencilla:

\begin{lstlisting}[language={[latex]tex}]
\begin{equation}
  \begin{matrix}
    a & b & c \\
    d & e & f \\
    g & h & i \\
  \end{matrix}
\end{equation}
\end{lstlisting}

\subsubsection{Sobre los paréntesis}\label{sec:parentesis}

Si no os apetece (como a mí) memorizar que el \lstinline!pmatrix! pone
un paréntesis y el \lstinline!vmatrix! no sé qué, podéis poner los
delimitadores vosotros según os parezca y usar siempre el entorno
\lstinline!matrix! (es lo que yo hago) pero hay que tener en cuenta una
cosa, en LaTeX hay dos tipos de paréntesis: los de tamaño fijo y los que
se adaptan al contenido. Los de tamaño fijo son tal cual el símbolo
según le damos en el teclado, los adaptativos son comandos formados por
\lstinline!\left! o \lstinline!\right!, según el lado, más el símbolo.
Por ejemplo, \lstinline!\left(! nos crea el paréntesis adaptativo del
lado izquierdo, \lstinline!\right]! el corchete adaptativo de la derecha
y así con todos. Los únicos un poco diferentes son los comandos para las
llaves, que requieren una barra de escape y son respectivamente
\lstinline!\left\{! y \lstinline!\right\}! Por ejemplo, para rodear la
matriz anterior con corchetes tendríamos que hacer lo siguiente:

\begin{lstlisting}[language={[latex]tex}]
\begin{equation}
  \left[
  \begin{matrix}
    a & b & c \\
    d & e & f \\
    g & h & i \\
  \end{matrix}
  \right]
\end{equation}
\end{lstlisting}

\section{Gestión del espacio}\label{sec:espacio}

Al igual que con el texto, LaTeX nos gestiona el espacio entre los
símbolos él solito. En general lo mejor es dejarle hacer, pero hay en
ocasiones en hay cosas que quedan \emph{feas} y hay que tocarlas un
poquito a mano. Los nazis del LaTeX nos dirán que no hay que hacer estas
cosas, que las decisiones de LaTeX deben ser respetadas. Yo no estoy de
acuerdo, la cuestión es que las ecuaciones queden a nuestro gusto. Para
ello utilizo estos dos chismes, aunque hay muchos más, que no son
específicos de las ecuaciones pero es donde suelen resultar más
necesarios:

\begin{itemize}
\item
  \lstinline!\,!: nos genera un espacio en blanco estrecho
\item
  \lstinline!~!: nos crea un
  \href{https://es.wikipedia.org/wiki/Espacio_duro}{\emph{espacio
  duro}}, es decir, un espacio que impide que se salte de línea en
  medio.
\end{itemize}

Como tampoco soy una sabia de la tipografía con estos dos me apaño, en
las referencias tenéis más y mejores explicaciones si os va el tema.

\section{Referencias cruzadas}\label{sec:refCruzadas}

Igual que las imágenes, las ecuaciones también se pueden referenciar
haciendo uso de los comandos \lstinline!\label! y \lstinline!\ref!. El
primero de ellos nos permite darle un nombre identificativo a una
ecuación y el segundo nos la cita. Al igual que ocurría con las figuras,
para poder añadir una etiqueta a una ecuación es necesario utilizar el
entorno \lstinline!equation!, no nos vale para las ecuaciones
\emph{inline}. Veamos cómo citaríamos la ecuación del primer ejemplo:

\begin{lstlisting}[language={[latex]tex}]
\begin{equation}
  e^{i\pi} + 1 = 0
  \label{eq:euler}
\end{equation}

Como vemos en la Ecuación \ref{eq:euler}
\end{lstlisting}

Que nos daría este resultado:

\begin{equation*} 
  e^{i\pi} + 1 = 0
\end{equation*}

Como vemos en la Ecuación 1

Añadir \lstinline!eq:! a la etiqueta no es necesario pero nos facilita
el trabajo al no tener las etiquetas para las figuras, las secciones y
demás elementos mezclados. También podemos definir un comando para que
nos añada la palabra \emph{Ecuación} al número. Os voy a decir cómo lo
haríamos aunque todavía no sepamos crear comandos para que veáis que es
sencillito\footnote{Creo que hay una manera mejor de hacer de definir
  este comando pero no me acuerdo y soy completamente incapaz de
  encontrarlo.}:

\begin{lstlisting}[language={[latex]tex}]
% Estructura \newcommand{\nombre}[nºargs]{Descripción}
\newcommand{\refeq}[1]{Ecuación~\ref{#1}}
\end{lstlisting}

Esto mismo lo consigue el comando
\href{https://en.wikibooks.org/wiki/LaTeX/Labels_and_Cross-referencing\#The_hyperref_package}{\lstinline!\\autoref!
del paquete \lstinline!hyperref!} con la ventaja de que nos pone la
palabra correcta en todos los casos, ya sean tablas, figuras o
ecuaciones sin necesidad de definir un comando para cada uno.

\section{Grupos de ecuaciones}\label{sec:gruposEc}

Otro tema interesante es poder escribir un grupo de ecuaciones que
comparta la misma etiqueta. Esto es posible (¡como todo en LaTeX!)
gracias a diferentes entornos aunque yo solo voy a hablar de mi
favorito: \lstinline!align! del paquete \lstinline!amsmath!. Nos permite
crear un sistema de ecuaciones que alineará según el símbolo que
marquemos con un ampersand. Por ejemplo, las 3 leyes de la termodinámica
quedarían así, alineadas según el símbolo de igual:

\begin{lstlisting}[language={[latex]tex}]
\begin{align}
  \Delta U &= Q -W \\
  \delta S &= T \mathrm{d}S \\
  S &=\mathrm{k_B}\ln\Omega
\end{align}
\end{lstlisting}

\begin{align*}
  \Delta U &= Q -W \\
  \delta S &= T \mathrm{d}S \\
  S &=\mathrm{k_B}\ln\Omega
\end{align*}

Al igual que hacíamos en el entorno \lstinline!equation!, con
\lstinline!align! también podemos añadir una etiqueta o usar el
asterisco para que no nos numere la ecuación.

\section{Formato}\label{sec:formato}

Evidentemente, LaTeX nos permite adaptar el formato de nuestras
ecuaciones a nuestros gustos o exigencias externas (véase formato de
revistas científicas, normas ISO \ldots{}). Un formato muy típico es el
siguiente:

\begin{itemize}
\item
  \emph{Cursiva para las variables}: LaTeX nos lo hace por defecto
\item
  \emph{Operadores y constantes rectos}: para los operadores del propio
  LaTeX como \lstinline!\sin! o \lstinline!\log! no tenemos que hacer
  nada, los endereza de por sí. Para el resto tenemos dos opciones: usar
  \lstinline!\mathrm! o definirlos como operadores en el preámbulo con
  \lstinline!\DeclareMathOperator! del paquete \lstinline!amsmath!. De
  esto último hablaremos más adelante, pero como sé que sois ansiosos os
  pongo cómo se haría:
\end{itemize}

\begin{lstlisting}[language={[latex]tex}]
\usepackage{amsmath}
\DeclareMathOperator{\comando}{descripción}
\end{lstlisting}

\begin{itemize}
\item
  \emph{Matrices y vectores en negrita}: para ello usaremos
  \lstinline!\mathbf! para las letras y \lstinline!\boldsymbol! para los
  símbolos o letras griegas.
\end{itemize}

Un ejemplo con todos ellos podría ser la definición de la matriz de
rigidez para el método de los elementos finitos (me sale el ingeniero
mecánico interior):

\begin{lstlisting}[language={[latex]tex}]
\begin{equation}
  \mathbf{K}=\int_V \mathbf{B}^\intercal \mathbf{D B}\mathrm{d}x\mathrm{d}y \mathrm{d}z
\end{equation}
\end{lstlisting}

que quedaría algo de este estilo:

\begin{equation*}
  \mathbf{K}=\int_V \mathbf{B}^\intercal\mathbf{D B}\mathrm{d}x \mathrm{d}y \mathrm{d}z
\end{equation*}

\section{Referencias}\label{sec:referencias}

\href{https://en.wikibooks.org/wiki/LaTeX/Mathematics}{\emph{LaTeX/Mathematics}
en WikiBooks}

\href{https://www.sharelatex.com/learn/List_of_Greek_letters_and_math_symbols}{\emph{List
of Greek letters and math symbols} en ShareLaTeX}

\href{http://latex.wikia.com/wiki/Matrix_environments}{\emph{Matrix
environments} en LaTeX Wiki}

\href{https://www.sharelatex.com/learn/Brackets_and_Parentheses}{\emph{Brackets
and Parentheses} en ShareLaTeX}

\href{https://www.sharelatex.com/learn/Operators}{\emph{Operators} en
ShareLaTeX}

\href{http://tex.stackexchange.com/questions/74353/what-commands-are-there-for-horizontal-spacing\#74354*}{\emph{What
commands are there for horizontal spacing?} en StackExchange}

\href{http://www.colorado.edu/physics/phys4610/phys4610_sp15/PHYS4610_sp15/Home_files/LaTeXSymbols.pdf}{\emph{Lista
de símbolos matemáticos} (pdf)}

\href{http://tex.stackexchange.com/questions/32100/what-does-each-ams-package-do}{\emph{What
does each AMS package do?} en StackExchange}

\href{http://moser-isi.ethz.ch/docs/typeset_equations.pdf}{\emph{How to
typeset equations in LaTeX} (pdf)}
