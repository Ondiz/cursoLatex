Vamos a hablar del caso en el que LaTeX da lo mejor de sí mismo: un
\textbf{documento científico}. Con esto me refiero a libros técnicos,
tesis, artículos\ldots{} ya me entendéis. No tienen por qué hablar
necesariamente de ciencia, si me centro más en ellos es porque es mi
campo.

La característica esencial de un documento científico es su
\emph{formato rígido} que en muchas ocasiones nos viene impuesto, ya sea
por una revista o universidad o por las propias costumbres de nuestra
disciplina. Por ejemplo, una tesis suele estar dividida en capítulos,
debe tener referencias bibliográficas con un estilo determinado y se
inicia con un índice general, uno de tablas y otro de figuras así como
con un glosario de términos. También suele llevar un resumen del
contenido al principio y se separa el trabajo adicional en apéndices.
Todo esto implica diferentes formatos para las partes (numeración de las
páginas y secciones, estilo de los encabezados\ldots{}) y cierta
planificación. Sin planificar se puede uno volver completamente tarumba,
lo digo con conocimiento de causa, el control de versiones me salvó más de
una vez de romper la tesis sin remedio.

Es por ello que primero veremos cómo organizar el documento en cuestión
y luego haremos un índice, un glosario y hasta una lista de
referencias.

\section{División del documento}

Vamos a empezar organizándonos. Lo primero que tenemos que hacer es
elegir el tipo de documento, empezando por preguntarnos si es un
documento corto o largo ya que el formato del documento varía en gran
medida según su longitud:

\begin{itemize}
\item
  \textbf{Documentos cortos}: para ellos usaremos la clase
  \lstinline!article! y sus derivadas, como \lstinline!scrartcl! de las
  \href{https://www.ctan.org/pkg/koma-script}{clases KOMA}. Como norma
  general este tipo de documentos tienen 5 niveles de títulos\footnote{Aprendimos
    a usar los diferentes niveles de título cuando escribimos nuestro
    \href{https://ondiz.github.io/cursoLatex/Contenido/03.DocumentoBasico.html}{primer
    documento}.} (\lstinline!section!, \lstinline!subsection!,
  \lstinline!subsubsection!, \lstinline!paragraph!,
  \lstinline!subparagraph!), con los tres primeros numerados, y no
  llevan portada.
\end{itemize}

\begin{itemize}
\item
  \textbf{Documentos largos}: en este caso nos convienen las clases
  \lstinline!book!, \lstinline!report!\footnote{Estas dos clases sirven
    para objetivos diferentes: \lstinline!report! está pensado para
    documentos no muy largos como un trabajo de clase o así y
    \lstinline!book! para libros verdaderos. Hay varias diferencias
    entre ellas, por ejemplo, la clase \lstinline!book! le da diferentes
    estilos a las páginas pares e impares por defecto mientras que
    \lstinline!report! solo lo hace si se lo pedimos; \lstinline!book!
    abre los capítulos siempre a la derecha aunque pare ello tenga que
    dejar una página en blanco, \lstinline!report! simplemente salta de
    página. Tenéis una comparación mejor
    \href{http://tex.stackexchange.com/questions/36988/ddg\#36989}{aquí}.}
  y sus derivadas como \lstinline!memoir!. Estos suelen tener 7 niveles
  de títulos (\lstinline!part!, \lstinline!chapter!,
  \lstinline!section!, \lstinline!subsection!,
  \lstinline!subsubsection!, \lstinline!paragraph!,
  \lstinline!subparagraph!) de los que se numeran los cuatro primeros.
  Los demás niveles tienen un estilo pero ni se numeran ni aparecen en
  el índice. A diferencia de los documentos cortos, estos llevan portada
  por defecto.
\end{itemize}

Para el caso de un documento largo, si estamos usando las clases
\lstinline!book!, \lstinline!memoir! o derivadas\footnote{\lstinline!report!
  carece de estos comandos el pobrecillo.}, aparte de los diferentes
niveles de títulos tenemos a nuestra disposición unos comandos para
identificar las distintas partes del documento y así darles estilos
específicos. Todos ellos afectan desde que los escribimos hasta que
aparece el siguiente:

\begin{itemize}
\item
  \lstinline!\frontmatter! identifica las páginas preliminares. En esta
  parte los capítulos no llevan número y las páginas se numeran en
  romano. Es donde suelen ir el índice o los agradecimientos.
\item
  \lstinline!\mainmatter! da inicio al cuerpo del documento. Se numeran
  los capítulos y las páginas con números arábigos (o como decían mis
  compañeros de clase \emph{números normales}).
\item
  \lstinline!\appendix! como su nombre indica, especifica dónde empiezan
  los apéndices. Aquí los capítulos se \emph{numeran} con letras pero se
  mantiene numeración de las páginas.
\item
  \lstinline!\backmatter! identifica las páginas finales. En esta parte
  los capítulos no se numeran y, como en los apéndices, se mantiene la
  numeración de las páginas. Nos vale, por ejemplo, para las referencias
  bibliográficas.
\end{itemize}

Por supuesto, todo lo que he comentado sobre los estilos se puede
modificar con relativa facilidad, pero sin tocar absolutamente nada
podemos conseguir un documento con bastante lógica.

Otro tema a tener en cuenta es la \textbf{división del contenido en
diferentes archivos}. Hasta ahora hemos escrito todo junto, pero cuando
estamos escribiendo un documento más o menos largo este sistema no es
muy apropiado. LaTeX nos permite escribir en diferentes archivos que
luego juntaremos para crear el documento final. De esta manera podemos
tener un archivo con la definición del documento y el preámbulo desde el
que llamaremos a otros archivos en los que nos ocuparemos del contenido
propiamente dicho.

Para ello tenemos dos opciones:

\begin{itemize}
\item
  \lstinline!\input{RUTA}! equivale a meter el contenido del archivo ahí
  mismo. Tiene la ventajas de que se puede usar en cualquier parte
  (¡incluido el preámbulo!) y que se puede anidar, es decir, el archivo
  \emph{B} que llamamos desde \emph{A} puede a su vez llamar al archivo
  \emph{C}.
\item
  \lstinline!\include{RUTA}! es un poco más complejo, digamos que nos
  permite \emph{compilar el documento a trozos} porque mantiene los
  números de página, secciones y demás como si compilásemos el documento
  completo aunque solo estemos compilando un pedacito\footnote{Esto
    ocurre porque escribe archivos auxiliares para cada archivo que
    llamamos y luego lee de ahí las referencias, números de página y
    otras cosas que varían. Más adelante hablaremos sobre archivos
    auxiliares, si tenéis ansias podéis leer
    \href{https://ondahostil.wordpress.com/2016/11/17/lo-que-he-aprendido-archivos-auxiliares-de-latex/}{esto
    que escribí hace un tiempo}.}. Es especialmente apropiado para
  capítulos ya que provoca un salto de página antes y después de incluir
  el contenido. Sus desventajas son que no se puede anidar y que no se
  debe usar en el preámbulo.
\end{itemize}

Vamos a ver un ejemplo de uso para entenderlo mejor:

\begin{lstlisting}[language={[latex]tex}]
\documentclass[a4paper,11pt]{book}

  % Cargamos el preámbulo que tenemos escrito en otro archivo
  \input{preámbulo}

  % Elegimos qué capítulos queremos que se compilen
  \includeonly{cap1,cap3} % sin espacios!

\begin{document}

  % Cargamos la portada desde la Carpeta contenido
  \input{Contenido/portada}

  % Cargamos los capítulos desde la carpeta Contenido.
  % Los capítulos capX.tex no llevan ni preámbulo ni definición del
  % documento, solo órdenes que irían tras \begin{document}

  \include{Contenido/cap1} % cargamos cap1.tex
  \include{Contenido/cap2} % cap2 no aparecerá en el resultado
  \include{Contenido/cap3} % mantiene la numeración como si cap2.tex estuviera incluido

\end{document}
\end{lstlisting}

Ya que tenemos comandos que nos permiten incorporar contenido desde
otros archivos podemos aprovechar para organizar nuestro material en
carpetas para el contenido, el estilo o las imágenes. Incluso cabe la
posibilidad de escribir las tablas largas en un archivo aparte y
llamarlas con \lstinline!\input{}! cuando corresponda. Cuando sepamos
más sobre los diferentes tipos de archivos os contaré cómo me organizo
yo.

\section{Índices}

Pasemos a hablar de los índices, uno de los motivos por los que amo
LaTeX. ¡Los índices de contenido, tablas, figuras o código se hacen
solos! Solamente tenemos que escribir el comando correspondiente y ya
está. La única cosa que tenemos que tener en cuenta es que los índices
no se incluyen por defecto en el índice de contenido, tenemos que
\emph{exigirle} a LaTeX que los añada.

Os pongo un ejemplo de cómo se crean los índices y cómo se incluyen en
el índice, creo que el funcionamiento es bastante claro:

\begin{lstlisting}[language={[latex]tex}]
% Índice de contenido
\addcontentsline{toc}{chapter}{Índice general}
\tableofcontents

% Índice de tablas
\addcontentsline{toc}{chapter}{Índice de tablas}
\listoftables

% Índice de figuras
\addcontentsline{toc}{chapter}{Índice de figuras}
\listoffigures
\end{lstlisting}

Tenemos que tener en cuenta que hay que compilar el documento dos veces
para que LaTeX fabrique los índices, la primera de ellas escribe los
archivos auxiliares necesarios y la segunda los lee y crea el índice en
concordancia.

Como todo en LaTeX, los índices se pueden personalizar. Por ejemplo,
para el caso del índice de contenido, se puede cambiar la profundidad de
los tres niveles que tiene por defecto al número que a nosotros nos
parezca mejor:

\begin{lstlisting}[language={[latex]tex}]
\setcounter{tocdepth}{NIVEL}
% NIVEL -1: part, 0: chapter, 1: section, etc.
\end{lstlisting}

También es interesante darle un título corto para los índices como
argumento opcional a \lstinline!\caption!, \lstinline!\chapter! y otros
comandos, por ejemplo:

\begin{lstlisting}[language={[latex]tex}]
\begin{figure}
  \caption[Título corto para el índice]{Título ultralargo que describe la figura en el documento}
    \includegraphics{RUTA}
\end{figure}
\end{lstlisting}

Para acabar con los índices una cosa bastante chula: si usamos el
paquete \lstinline!hyperref! los elementos de los índices se vuelven
\emph{clicables}. Hasta podemos poner los enlaces de colorines, en mi
tesis, por ejemplo, eran rosas gracias a esta línea:

\begin{lstlisting}[language={[latex]tex}]
\usepackage[colorlinks=true,linkcolor=magenta]{hyperref}
\end{lstlisting}

Dicen en el manual de \lstinline!hyperref! que es mejor que sea el
último paquete que cargamos por si acaso colisiona con algún otro.

\section{Glosario y unidades}

Otro tema que nos interesa a nosotros los científicos es poder hacer una
lista de símbolos, letras griegas o acrónimos con facilidad. Y facilidad
no es coger un documento científico de 350 páginas y apuntar en un
papelito todos los símbolos que hemos utilizado con su respectiva
definición\footnote{Yo me sé de uno que casi gasta el alfabeto griego en
  su tesis y tenía subíndices y superíndices simultáneamente para hacer
  referencia a diferentes variables. Le mando un beso desde aquí.}.

Para el tema de los glosarios, nomenclaturas o listas de
símbolos\footnote{Voy a usar \emph{lista de símbolos} y \emph{glosario}
  indistintamente para referirme a \emph{una lista de cosas con su
  respectiva definición que ponemos al inicio del documento para ayudar
  al lector a entender lo que hemos escrito}} es muy importante la
planificación. \emph{Muy importante}. Lo mejor es configurar todo antes
de empezar a escribir, así aprovechamos las ventajas que tiene usar un
paquete de este estilo y de paso no nos volvemos locos. Por cierto, digo
\emph{un paquete} porque hay unos cuantos con este propósito, aunque
todos ellos tienen dos características en común:

\begin{itemize}
\item
  \textbf{Declaramos los símbolos al principio}. De este modo
  garantizamos la coherencia y no nos llevamos la sorpresa de que le
  hemos llamado \emph{I} a la corriente en el capítulo 3 y \emph{J} en
  el 4. Además, al cambiar la declaración se modifican los símbolos
  correspondientes en todo el documento.
\item
  \textbf{La lista de símbolos se crea automáticamente}. Dependiendo del
  paquete tenemos la opción de crear listas de símbolos, letras griegas
  y acrónimos separadas o no, pero en todos los casos la lista se genera
  automáticamente de manera similar al índice de contenido y solo
  incluye los símbolos que hemos usado en el documento. Esto último
  igual parece ridículo así a primera vista pero nos permite tener
  declarados un montón de símbolos en un archivo aparte y tener la
  certeza de que solo se incluirán los que aparecen en el documento en
  el que estamos trabajando.
\end{itemize}

Veamos qué opciones tenemos. Por una parte están los que usan el
programa
\href{http://tex.loria.fr/bibdex/makeindex.pdf}{\lstinline!makeindex!}
lo que implica un paso extra a la hora de compilar:
\href{http://www.ctan.org/pkg/nomencl}{\lstinline!nomencl!} y
\href{http://www.ctan.org/tex-archive/macros/latex/contrib/glossaries/}{\lstinline!glossaries!},
heredero del antiguo
\href{http://www.ctan.org/pkg/glossary}{\lstinline!glossary!} y que
tiene un manual de 250 páginas. Por otro, tenemos al mucho más modesto
\href{https://www.ctan.org/pkg/listofsymbols}{\lstinline!listofsymbols!}
que tiene menores capacidades pero no nos dificulta la compilación.

Cuando escribí la tesis no tenía ganas de liarme la manta así que me
quedé con \lstinline!listofsymbols! que era sencillito y eficaz. ¡Al
final hasta me atreví a \href{https://gitlab.com/Ondiz/los}{añadirle
funcionalidades} con mis limitadas habilidades! Os dejo un ejemplo de
uso de \lstinline!listofsymbols! que creo que se entiende por sí solo:

\begin{lstlisting}[language={[latex]tex}]
\documentclass{article}

  \usepackage[draft]{listofsymbols} % cambiar 'draft' por 'final' en el definitivo
  \begin{document}

    % Definición de símbolos [Definición]{Nombre que usaremos}[Símbolo]
    \opensymdef
    \newsym[Energía]{symE}{E}
    \newsym[Masa]{symm}{m}
    \newsym[Velocidad de la luz]{symc}{c}
    % Hay que usar \ensuremath en los símbolos matemáticos
    \newsym[Longitud de onda]{symlam}{\ensuremath{\lambda}}
    \closesymdef

    % Crear lista de símbolos
    \listofsymbols

    % Uso en ecuaciones
    \begin{equation}
      \symE=\symm \symc^2
    \end{equation}

    % Uso en texto
    donde \symE es la energía \ldots

\end{document}
\end{lstlisting}

En lo que respecta a \lstinline!nomencl! y a \lstinline!glossaries!,
aunque me parecen interesantes nunca los he usado así que poco más os
puedo decir aparte de este pequeño \emph{estado del arte}. En las
referencias tenéis más información si os animáis a usarlos.

Algo parecido nos pasa con las unidades, por eso voy a hablar de ellas
en este mismo apartado. En este caso se une al tema de la coherencia la
tipografía, una de las mayores preocupaciones de los usuarios de LaTeX
del mundo\footnote{Hay hasta una
  \href{https://nickhigham.wordpress.com/2016/01/28/typesetting-mathematics-according-to-the-iso-standard/}{norma
  ISO} sobre cómo dar formato a las constantes, variables, unidades y
  demás. Y yo he visto a profesores criticar a gente en defensas de
  proyectos \emph{porque no han puesto recto el símbolo de la derivada}.
  Tiene que haber gente para todo.}.

Si nuestro principal problema es que las unidades tengan buena pinta,
tenemos el paquete
\href{http://www.ctan.org/tex-archive/macros/latex/contrib/units/}{\lstinline!units!}
que se ocupa de que nuestras unidades sean tipográficamente correctas.
Gracias a él las unidades tendrán el mismo estilo que el resto del texto
en el modo texto y serán rectas en el modo matemático; no habrá nunca un
salto de línea antes de la unidad, y habrá solo un \emph{espacio
estrecho} (\lstinline!\,!) entre el valor y la unidad. Incluso nos crea
una fracciones de lo más cucas gracias al paquete
\href{http://ctan.org/pkg/nicefrac}{\lstinline!nicefrac!}. Pero la
verdad es que lo uso porque tiene uno de los
\href{http://osl.ugr.es/CTAN/macros/latex/contrib/units/units.pdf}{manuales}
más geniales que he visto yo nunca.

Simplemente lo cargamos en el preámbulo:

\begin{lstlisting}[language={[latex]tex}]
\usepackage{units} % Opción 'ugly' para fracciones de texto normal
\end{lstlisting}

Y usamos los comandos \lstinline!\unit[VALOR]{UNIDAD}! y
\lstinline!\unitfrac[VALOR]{NUM}{DEN}! tanto en el modo texto como
dentro de ecuaciones:

\begin{lstlisting}[language={[latex]tex}]
\unit[3]{m}

$\unitfrac[4]{m}{s}$
\end{lstlisting}

Si aparte de unidades bonitas queremos garantizar la coherencia en todo
el texto tenemos el mucho más potente (y con manual mucho menos
gracioso) paquete
\href{http://ctan.org/pkg/siunitx}{\lstinline!SIunits!}. El enfoque de
\lstinline!SIunits! para las unidades es similar al de los paquetes para
hacer glosarios y al que tiene el propio LaTeX para las ecuaciones: usar
comandos para las unidades en lugar de texto normal. No voy a entrar en
mucho detalle porque es un paquete bastante complejo, pero básicamente
nos da unos comandos para dar formato a números, unidades y números con
unidades:

\begin{lstlisting}[language={[latex]tex}]
% Formato de número
\num[OPCIONES]{NÚMERO}

% Formato de unidad
\si[OPCIONES]{UNIDAD}

% Formato de valor con unidad
\SI[OPCIONES]{VALOR}{UNIDAD}
\end{lstlisting}

Podemos o bien escribir las unidades nosotros mismos o utilizar el
comando correspondiente:

\begin{lstlisting}[language={[latex]tex}]
\si{kg}

\si{\kilogram}
\end{lstlisting}

Es interesante porque nos permite representar las fracciones como
exponentes negativos y los prefijos como \emph{kilo} o \emph{mili} como
potencias de 10:

\begin{lstlisting}[language={[latex]tex}]
\SI{10}{\kilo\metre\per\hour} % 10 km h^{-1}
\SI[per-mode = fraction]{10}{\kilo\metre\per\hour} % 10 km/h
\SI[prefixes-as-symbols=false]{10}{\kilo\metre\per\hour}  % 10 10^3 m h^{-1}
\end{lstlisting}

Todo esto lo podemos configurar en el preámbulo para no tener que
escribir tantísimo, claro. Muchas de las unidades tienen además nombres
cortos para que no sean tan cansinas de definir.

\section{Referencias bibliográficas}

Ahora toca mi parte favorita en lo que respecta a LaTeX y los documentos
científicos: el manejo de las referencias bibliográficas. Para ello
usaremos un programa diferente pero que va integrado en las
distribuciones de LaTeX: el gestor de referencias bibliográficas
\href{http://www.bibtex.org/}{BibTeX}.

BibTeX nos permite tener una base de datos de documentos diversos que
podemos citar en el texto y que luego muy amablemente nos listará donde
le pidamos con nuestro estilo favorito. Lo mejor es que en esa lista
solo aparecerán los elementos que hemos citado y en el orden que
queramos independientemente del orden en el que aparezcan en la base de
datos.

Además, como no era suficientemente genial, tiene dos ventajas extra:
todo va en texto plano y se pueden usar programas libres para generar la
base de datos.

Antes de nada una cosilla, usar BibTeX como yo lo uso es la manera más
sencilla pero también hay otros paquetes más avanzados como
\href{http://ctan.org/pkg/natbib}{\lstinline!natbib!} y
\href{http://ctan.org/pkg/biblatex}{\lstinline!biblatex!} e incluso
otros programas como
\href{http://biblatex-biber.sourceforge.net/}{Biber}, especialmente
desarrollado para trabajar con \lstinline!biblatex!. Aquí me pasa como
con \lstinline!makeindex!, se que existen y poco más. En cualquier caso
os dejo algunos enlaces en las referencias para que os modernicéis, yo
ya no puedo que soy vieja.

Yendo al grano, para mi manera de gestionar la bibliografía necesitamos
lo siguiente:

\begin{itemize}
\item
  \textbf{Una bibliografía}, evidentemente, o bien la tenemos en un
  archivo \lstinline!.bib! o la definimos en el propio documento.
\item
  \textbf{Un estilo de cita} que vendrá definido en un archivo
  \lstinline!.bst!, la propia distribución de
  \href{https://www.sharelatex.com/learn/Bibtex_bibliography_styles}{LaTeX
  trae unos pocos} pero podemos descargarnos uno que nos interese de por
  ahí (por ejemplo, cuando una revista científica nos exige citar de una
  manera determinada) e incluso
  \href{http://ctan.org/pkg/custom-bib}{crear uno nosotros}.
\item
  Si vamos a tener nuestra base de datos en un \lstinline!.bib! nos
  vendrá bien \textbf{programa de gestión bibliográfica}. Puede ser un
  programa específico como los libres
  \href{http://www.jabref.org/}{Jabref} o
  \href{https://www.zotero.org/}{Zotero} o un editor de uso general como
  \href{https://nickhigham.wordpress.com/2016/01/06/managing-bibtex-files-with-emacs/}{Emacs}.
\item
  \textbf{Compilar 4 veces} en una secuencia \lstinline!pdflatex!,
  \lstinline!bibtex!, \lstinline!pdflatex!,
  \lstinline!pdflatex!\footnote{El proceso que se sigue es algo así: el
    primer comando escribe las claves de las referencias en el archivo
    auxiliar. A continuación \lstinline!bibtex! lee esas claves junto
    con el \lstinline!.bib! y el \lstinline!.bst! y crea la bibliografía
    en un archivo \lstinline!.bbl!. El segundo \lstinline!latex! escribe
    las referencias cruzadas en el \lstinline!aux!. Por fin, el último
    escribe las referencias cruzadas en el archivo final.}. He puesto
  \lstinline!pdflatex! pero ya sabemos que puede ser perfectamente
  \lstinline!xelatex! o simplemente \lstinline!latex!. Si estamos usando
  un IDE se hará automáticamente.
\end{itemize}

Vamos a hablar un poco sobre la base de datos bibliográfica. Para mí lo
más simple es usar un programa con su GUI y guardar todo un
\lstinline!.bib!. Este archivo no deja de ser texto plano, de hecho, si
lo abrimos con un editor de texto normal veremos que cada una de las
entradas de nuestra base de datos tiene una pinta como esta:

\begin{lstlisting}[language={[latex]tex}]
@Book{Zarraga2017,
  title={Curso no convencional de LaTeX},
  author={Zarraga, Ondiz},
  year={2017},
  note=\url{https://ondiz.github.io/cursoLatex/}
}
\end{lstlisting}

No voy a hablar sobre ninguno de estos programas aquí por dos motivos:
ya he escrito suficiente tocho y sus manuales están muy bien.

Si no nos gusta usar otro programa, siempre podemos definir la
bibliografía directamente en LaTeX:

\begin{lstlisting}[language={[latex]tex}]
\begin{thebibliography}{9} % Menos de 9 entradas

\bibitem{Zarraga2017}
  Ondiz Zarraga,
  \emph{Curso no convencional de LaTeX},
  2017.

\end{thebibliography}
\end{lstlisting}

En estos dos ejemplos \lstinline!Zarraga2017! es la \emph{clave} de la
entrada bibliográfica y es lo que usaremos para citar ese trabajo en
nuestro documento mediante el comando \lstinline!\cite{CLAVE}!:

\begin{lstlisting}[language={[latex]tex}]
Tal y como dice \cite{Zarraga2017}
\end{lstlisting}

Si hemos escrito nosotros la bibliografía a mano con
\lstinline!thebibliography!, en el documento nos sustituirá
\lstinline!\cite{Zarraga2017}! por un número. Si estamos usando un
archivo \lstinline!.bib!, en cambio, tenemos ventajas añadidas ya que
podemos pedir un estilo de cita y de referencia bibliográfica concreto.
Este estilo está definido en un archivo \lstinline!.bst!, tenéis
explicados los que BibTeX tiene por defecto
\href{https://www.sharelatex.com/learn/Bibtex_bibliography_styles}{aquí}.
En definitiva, necesitamos estas dos líneas:

\begin{lstlisting}[language={[latex]tex}]
% Definimos el estilo bibliográfico
\bibliographystyle{plain}

% Creamos bibliografía a partir de referencias.bib
\bibliography{referencias}
\end{lstlisting}

Vamos a dejar aquí la bibliografía, es un tema que da para mucho. Espero
que esto os sirva como introducción, hay más información en las
referencias.

\section{Recapitulación}

Hemos visto que la planificación es fundamental a la hora de escribir un
documento científico. La planificación se extiende tanto a la
organización de los ficheros como a la propia escritura del documento,
ya que elegir una manera de hacer las cosas antes de hacerlas nos ayuda
luego a la hora de trabajar. Por eso yo me haría unas pocas preguntas
antes de escribir la primera letra:

\begin{itemize}
\item
  ¿Qué tipo de documento nos conviene? ¿Hay alguna clase que nos
  facilite la labor?
\item
  ¿Necesitamos crear una lista de símbolos? ¿Vamos a hacerla a mano o es
  preferible usar un paquete? En este último caso, ¿usamos uno
  sencillito o incluimos una etapa extra de compilación para ganar en
  funcionalidad?
\item
  ¿Podría ser interesante un paquete para tratar las unidades? ¿Queremos
  solo que sean bonitas o también cierta coherencia?
\item
  Si nuestro documento tiene referencias bibliográficas ¿cómo vamos a
  crear la base de datos? ¿Qué programa vamos a utilizar?
\end{itemize}

También hemos aprendido que una buena idea para escribir un documento
largo es dividirlo en un archivo en el que se define su esqueleto y
otros en los que está el contenido propiamente dicho. El esqueleto
tendrá una pinta similar a esta:

\begin{lstlisting}[language={[latex]tex}]
% Definición del documento
\documentclass[opciones]{tipo}

% Paquetes
\usepackage[opc]{paquete}

% Datos
\title{título}
\author{autor}
\includeonly{cap1}

% Inicio
\begin{document}
  % Título e índices
  \maketitle

  \frontmatter
  \tableofcontents
  \listoftables
  \listoffigures

  % Contenido
  \mainmatter
  \include{cap1}
  \include{cap2}

  % Apéndices
  \appendix
  \include{ap1}

  % Referencias bibliográficas
  \backmatter
  \bibliographystyle{plain}
  \bibliography{referencias}
  % Fin
\end{document}
\end{lstlisting}

En definitiva, con este fascículo del curso he intentado hablar un poco
de diferentes paquetes y herramientas que tenemos a nuestra disposición
a la hora de escribir un documento científico. Me habré dejado cosas en
el tintero que espero recuperar en el futuro próximo. De momento ahí
están las referencias.

\section{Referencias}\label{referencias}

\href{http://texblog.org/2007/07/09/documentclassbook-report-article-or-letter/}{\emph{\lstinline!\\documentclass\{book, report,
article or letter\}!} en texblog}

\href{http://tex.stackexchange.com/questions/36988/regarding-the-book-report-and-article-document-classes-what-are-the-mai\#36989}{\emph{Regarding
the \lstinline!book!, \lstinline!report!, and \lstinline!article!
document classes: what are the main differences?} en TexExchange}

\href{http://web.science.mq.edu.au/~rdale/resources/writingnotes/latexstruct.html}{\emph{LaTeX
Notes: Structuring Large Documents}}

\href{http://tex.stackexchange.com/questions/20538/what-is-the-right-order-when-using-frontmatter-tableofcontents-mainmatter\#20547}{\emph{What
is the right order when using \lstinline!\\frontmatter!,
\lstinline!\\tableofcontents!, \lstinline!\\mainmatter!,
\lstinline!\\part!, \lstinline!\\chapter!, \lstinline!\\backmatter!,\lstinline!\\appendix! etc?} en TexExchange}

\href{http://tex.stackexchange.com/questions/246/when-should-i-use-input-vs-include\#250}{\emph{When
should I use \lstinline!\\input! vs. \lstinline!\\include!?} en TexExchange}

\href{http://texblog.org/2011/09/09/10-ways-to-customize-tocloflot/}{\emph{10
ways to customize toc/lof/lot}}

\href{http://get-software.net/macros/latex/contrib/glossaries/glossariesbegin.pdf}{\emph{The
glossaries package v4.25: a guide for beginners} (pdf)}

\href{http://texblog.org/2014/01/15/glossary-and-list-of-acronyms-with-latex/}{\emph{LaTeX
glossary and list of acronyms}}

\href{https://www.sharelatex.com/learn/Nomenclatures}{\emph{Nomenclatures}
en ShareLaTeX}

\href{http://www.math.illinois.edu/~ajh/tex/bibliographies.html}{\emph{LaTeX
tips: Bibliographies}}

\href{https://www.latex-tutorial.com/tutorials/beginners/latex-bibtex/}{\emph{Bibliography
in LaTeX with Bibtex/Biblatex}}

\href{ftp://ftp.dante.de/tex-archive/info/bibtex/tamethebeast/ttb_en.pdf}{\emph{Tame
the BeaST} (pdf)}

\href{https://www.sharelatex.com/learn/Bibliography_management_with_natbib}{\emph{Bibliography
management with \lstinline!natbib!} en ShareLaTeX}

\href{http://tex.stackexchange.com/questions/13509/biblatex-in-a-nutshell-for-beginners}{\emph{\lstinline!biblatex!
in a nutshell (for beginners)} en TeXExchange}

\href{http://www.colorado.edu/physics/phys4610/phys4610_sp13/bibtex_guide.pdf}{\emph{A
BibTeX Guide via Examples} (pdf)}

\href{http://debibify.dorian-depriester.fr/}{\emph{Debibify}, un
colección de estilos de cita}
