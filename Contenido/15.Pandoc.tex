Allá por los inicios del cursete de LaTeX hablé de Pandoc, una
herramienta que transforma documentos de un formato a otro fácilmente y,
sobre todo, \emph{bien}. Comenté que Pandoc nos permitía escribir en
Markdown y obtener un \emph{pdf} de calidad porque por el medio usaba
LaTeX. Hoy vamos a ver cómo se hace esto.

\emph{Si ya sabemos LaTeX para que vamos a aprender Pandoc ahora}, me
diréis. Pues porque tiene unas pocas ventajas que paso a listar:

\begin{itemize}
\item
  El documento es \textbf{legible} incluso antes de compilar.
\item
  Podemos usar \textbf{cualquier editor} de texto, cosa que también
  ocurre con LaTeX pero que exige un proceso de compilación más
  complejo.
\item
  \textbf{Tecleamos menos}, ya que Markdown es un lenguaje mucho más
  escueto que LaTeX. Podemos suplir las carencias de Markdown mezclando
  LaTeX alegremente por ahí y usando filtros.
\item
  \textbf{No nos deja archivos auxiliares} por ahí, los borra una vez
  hecha la transformación. Solo nos quedan el Markdown inicial y el
  \emph{pdf} final.
\item
  Tenemos la opción de obtener \textbf{diferentes formatos finales}, lo
  que nos permite, por ejemplo, montar una web y sacamos un \emph{pdf} a
  partir de los mismos archivos.
\end{itemize}

Como todo no puede ser bueno, también tiene unas pocas desventajas:

\begin{itemize}
\item
  Necesitamos más archivos, si es un documento complejo nos vendrá bien
  un Makefile.
\item
  Instalarlo puede ser un poco lío, sobre todo los filtros.
\item
  Tenemos que saber LaTeX (pero ya sabemos ¿no?), Markdown (que se
  aprende en diez minutos) y Pandoc (otros diez minutos).
\end{itemize}

Yo escribí mi tesis doctoral así, primero porque estoy loca y segundo
porque me hago vieja y mis manos necesitan un descanso del ordenador. He
de reconocer que ese es un caso muy extremo, pero este sistema es muy
práctico para tomar notas rápidas y que el resultado final tenga buena
pinta.

\section{Instalación y uso}

Hay varias maneras de instalar Pandoc, desde la sencillísima de tirar de
repositorios o del
\href{https://github.com/jgm/pandoc/releases/tag/1.19.2.1}{paquete de
Debian o Windows correspondiente} a usar \lstinline!cabal!, el gestor de
paquetes de Haskell. Hay que tener en cuenta que la versión de los repos
suele ser bastante antigua\footnote{Haciendo
  \lstinline!apt-cache search pandoc! en Ubuntu 16.04 LTS he visto que
  lleva un Pandoc 1.16} y que si instalamos desde un paquete tendremos
que reinstalar si queremos un Pandoc actualizado a tope.

Por esos motivos y porque siempre tengo la idea de aprender Haskell y
luego nunca lo hago, yo instalé Pandoc y dos filtros que utilizo mucho
mediante \lstinline!cabal!\footnote{Recordad añadir
  \lstinline!.cabal/bin! al \emph{path} si queréis llamar a Pandoc y sus
  filtros desde cualquier sitio sin dar la ruta completa.}:

\begin{lstlisting}[language=bash]
sudo apt-get install haskell-platform   # O equivalente según sistema operativo
cabal update                            # Actualizar lista de paquetes
cabal install pandoc                    # Instalar pandoc
cabal install pandoc-citeproc           # Instalar filtro de bibliografía
cabal install pandoc-crossref           # Instalar filtro de referencias cruzadas
\end{lstlisting}

Una vez que lo tenemos instalado, nos toca aprender a usarlo. Lo más
raro de Pandoc es que \textbf{se usa desde la terminal}, nada que
nosotros aguerridos \emph{latexianos} no podamos dominar. En el caso más
simple no tenemos más que darle el nombre del archivo de entrada y el de
salida con su respectiva extensión:

\begin{lstlisting}[language=bash]
pandoc INPUT.ext -o OUTPUT.ext
\end{lstlisting}

Ahora nos vamos a centrar en el caso de crear un \emph{pdf} a partir de
uno o varios archivos escritos en Markdown. Aparte de la opción
\lstinline!-o! que acabamos de ver hay otras que nos pueden resultar
interesantes:

\begin{itemize}
\item
  \lstinline!--latex-engine = COMPILADOR!: nos permite elegir el
  compilador de LaTeX. Como sabemos, el compilador elegido afecta a cómo
  se establecen el idioma y las fuentes.
\item
  \lstinline!--variable!, \lstinline!-V!: controla el comportamiento de
  la plantilla que LaTeX utiliza para pasar de Markdown a \emph{pdf}
  asignando valores a las diferentes variables. Un ejemplo es establecer
  el tipo de documento, que se hace con
  \lstinline!-V documentclass=book!. Estas variables se pueden
  introducir mediante un bloque YAML en el propio archivo Markdown o en
  un archivo \emph{.yaml} como veremos un poco más abajo.
\item
  \lstinline!--number-sections!, \lstinline!-N!: numera las secciones.
\item
  \lstinline!--top-level-division=[part|chapter|section]!: indica si la
  división de mayor jerarquía es la parte, el capítulo o la sección. Se
  establece automáticamente si definimos el tipo de documento.
\item
  \lstinline!--table-of-contents!, \lstinline!--toc!: añade un índice al
  documento. Esta opción también puede controlarse desde el bloque YAML,
  la pongo como ejemplo de la flexibilidad que tiene el programa.
\item
  \lstinline!--template=RUTA!: indica el camino a la plantilla que debe
  seguir para dar formato al documento.
\item
  \lstinline!--filter FILTRO!: añade funcionalidades adicionales
  incluidas en un filtro. Los filtros que he mencionado en la
  instalación, \lstinline!pandoc-citeproc! y
  \lstinline!pandoc-crossref!, sirven respectivamente para gestionar las
  referencias bibliográficas y las referencias cruzadas sin tener que
  recurrir a sintaxis de LaTeX.
\end{itemize}

Todas estas cosas parecen complejas pero en la mayor parte de los casos
usaremos el comando simple que mostraba al principio, montaremos nuestro
contenido en un plis con Markdown y tendremos un documento final decente
gracias al LaTeX intermedio.

\section{¿Qué necesitamos?}

Vamos a ver más en detalle qué nos hace falta para crear un \emph{pdf} a
partir de un Markdown. En esencia necesitamos:

\begin{itemize}
\item
  Archivos escritos en Mardown con extensión \emph{md}
\item
  Pandoc y una distribución de LaTeX
\item
  Una consola
\item
  Si vamos a tener referencias bibliográficas, un archivo con la
  bibliografía y un estilo de bibliografía \emph{csl}. Nuestro amado
  \emph{.bib} nos vale perfectamente para la primera labor y tenemos
  estilos \emph{csl} a patadas en
  \href{https://www.zotero.org/styles}{la web de Zotero}.
\item
  Una plantilla si la que usa por defecto no nos gusta
\end{itemize}

Para hacernos la vida más fácil nos conviene:

\begin{itemize}
\item
  Un archivo YAML para las variables
\item
  Un Makefile y \lstinline!make!
\end{itemize}

No os preocupéis, enseguida diseccionamos las cosas nuevas: el Markdown,
la plantilla, el YAML y el Makefile.

\subsection{Markdown sabor Pandoc}

Markdown es un viejo conocido de los que escribimos \emph{en el
Internet}, especialmente de los que usamos GitHub. Pandoc tiene un
Markdown especial con más funcionalidades que el
\href{https://daringfireball.net/projects/markdown/}{Markdown estricto},
como notas al pie o la posibilidad de cambiar el tamaño de las imágenes,
esto último en sus versiones más modernas. Va un resumen rápido:

\begin{lstlisting}
# Título de primer nivel

## Título de segundo nivel

Lista numerada:

1. Ítem
1. *cursiva* o _cursiva_
1. **negrita**

Lista sin numerar:

* [enlace](url) o [enlace] o [enlace][ref]
* Texto[^nota]

[^nota]: Texto de la nota al pie
[enlace]: url
[ref]: url

### Título de tercer nivel

Imagen

![Texto alternativo](ruta a la imagen){width=porcentaje}

Tabla

------------
  x      y
----- ------
 0.1   1.3
------------

Table: Aquí va el pie de tabla

$Ecuación inline$

$$Ecuación con línea propia$$

`Código inline`

Bloque de código

```
haskell
-- Fibonacci!
  fib = 0 : 1 : zipWith (+) fib (tail fib)
```
\end{lstlisting}

Además, los filtros que mencionaba tienen una sintaxis propia para las
referencias cruzadas y bibliografía. Van unos ejemplos:

\begin{lstlisting}
# Sección {#sec:etiqueta}

Como dicen en [@claveBibliográfica] o incluso en @claveBibliográfica

![Pie de figura](ruta){#fig:etiqueta}

Tal y como se ve en @fig:etiqueta

$$ ecuación $$ {#eq:etiqueta}

La ecuación @eq:etiqueta
\end{lstlisting}

\subsection{Plantilla personalizada}

Pandoc realiza la conversión de archivos mediante una plantilla para
cada tipo de formato. Hemos dicho que para pasar de Markdown a
\emph{pdf} utiliza LaTeX, podemos extraer la plantilla correspondiente
haciendo:

\begin{lstlisting}[language=bash]
pandoc -D latex > plantilla.latex
\end{lstlisting}

Al abrirla veremos que contiene LaTeX de toda la vida y que, por lo
tanto, sabemos adaptarla a nuestro gusto.

Lo único así raro son las cosas que van entre símbolos de dólar y que no
son ecuaciones sino \emph{variables}. Entre ellas la más importante es
\lstinline!$body$!, donde Pandoc mete el contenido de nuestros Markdowns
una vez los ha convertido a LaTeX.

\subsection{Metadatos YAML}

Todas las variables que aparecen en la plantilla se sustituyen con los
valores o \emph{metadatos} que enviamos con la opción \lstinline!-V!, en
un bloque YAML dentro de un archivo Markown o en un archivo YAML aparte.

Un bloque YAML real tiene una pinta como esta:

\begin{lstlisting}
---
# Datos
title: Título
author: Nombre
lang: es

# Control
toc: True

# Formato
documentclass: book
geometry:
- top=1in
- bottom=1in
- right=0.5in
- left=1.5in
mainfont: LiberationSans
fontsize: 12pt

# Bibliografía
bibliography: referencias.bib
csl: formato.csl
link-citations: true
---
\end{lstlisting}

Se puede añadir este bloque al inicio de nuestro Markdown o guardarlo en
un archivo aparte que incluiremos como entrada para Pandoc:

\begin{lstlisting}[language=bash]
pandoc datos.yaml input.md -o output.pdf
\end{lstlisting}

Es importante que aunque tengamos los datos en un archivo aparte
mantengamos los guiones alrededor, ya que como Pandoc concatena todos
los archivos de entrada no sabrá que es un bloque YAML sin ellos.

\subsection{Makefile}

Si juntamos todo lo visto hasta ahora nos damos cuenta de que la orden
para invocar a Pandoc puede ser muy larga. Ahí es donde nos resulta útil
un \emph{Makefile}, un archivo que contiene las instrucciones para
compilar determinada \emph{cosa}, en nuestro caso para convertir los
Markdowns en \emph{pdfs}.

Yo no os voy a explicar cómo hacer un \emph{Makefile}, básicamente
porque no sé, os dejo un ejemplo con varios archivos de entrada y
filtros que podéis adaptar y un par de referencias sobre el tema. Lo
único que sé es que debe tener esta estructura:

\begin{lstlisting}[language=make]
variable = contenido

objetivo:
    órdenes indentadas un tabulador
    $(referencia a variable)
\end{lstlisting}

Mi \emph{Makefile} habitual suele parecerse a este,
\href{https://gist.github.com/ekaitz-zarraga/f90f4c9c46a394e2048a\#file-makefile}{la
gente que sabe de ordenadores lo hará mucho mejor}:

\begin{lstlisting}[language=make]
# Archivos markdown

FILES = INPUT_1.md INPUT_2.md

# Creación de pdf
# - La barra sirve para partir la orden
# - El filtro crossref debe ir antes de citeproc
# - Opción -N para numerar secciones

all:
    pandoc \
    --filter pandoc-crossref \
    --filter pandoc-citeproc \
    --template=./PLANTILLA.latex \
    -N \
    $(FILES) METADATOS.yaml \
    -f markdown -o OUTPUT.pdf

# Eliminar output

clean:
    rm OUTPUT.pdf

# Por si hay algún archivo que se llame clean

.PHONY: clean
\end{lstlisting}

Ahora solo nos hace falta llamar a \lstinline!make!:

\begin{lstlisting}[language=bash]
make       # Crear el pdf
make clean # Borrar el pdf
\end{lstlisting}

\section{Reflexión final}

En este capítulo he intentado dar a conocer otra herramienta relacionada
con LaTeX y que me parece tremendamente útil. No creo que sea un
sustituto para el LaTeX puro y duro pero me parece interesante para
escribir documentos rápido y que tengan una apariencia elegante. Otro
uso a tener en cuenta es obtener salidas variadas a partir de nuestro
LaTeX, que sé yo, para los que tengáis jefes raros que solo lean
documentos \emph{en Word}.

Como siempre, yo intento dar una visión general del tema, para aprender
más os dejo las referencias.

\section{Referencias}

\href{http://pandoc.org/MANUAL.html}{Manual de Pandoc}

\href{http://programminghistorian.org/lessons/sustainable-authorship-in-plain-text-using-pandoc-and-markdown}{\emph{Sustainable
Authorship in Plain Text using Pandoc and Markdown}}

\href{http://pandoc.org/MANUAL.html\#pandocs-markdown}{\emph{Pandoc's
markdown}}

\href{https://github.com/jgm/pandoc-citeproc/blob/master/man/pandoc-citeproc.1.md}{\lstinline!pandoc-citeproc!}

\href{https://github.com/lierdakil/pandoc-crossref}{\lstinline!pandoc-crossref!}

\href{http://servbiob.inf.um.es/proyectos/public/attachments/download/8/texput.pdf}{\emph{Sinopsis
del formato markdown para pandoc} (pdf)}

\href{https://www.zotero.org/styles}{\emph{Zotero Style Repository}}

\href{https://es.wikipedia.org/wiki/Make}{\lstinline!make! en la
Wikipedia}

\href{https://pfctelepathy.wordpress.com/2015/05/11/intro-compilacion-linkado-y-makefiles-en-c/}{\emph{Compilación,
Linkado y Makefiles en C} en Free Hacks!}
