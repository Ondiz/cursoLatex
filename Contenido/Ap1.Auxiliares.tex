Os habréis fijado como yo en que cuando compilamos LaTeX desde un IDE se
generan miles y miles de archivos auxiliares al darle a que genere el
documento. Vamos a ver si conseguimos clarificar un poco que para qué
sirven algunos de ellos.

Aquí tenemos un lista de los archivos auxiliares más típicos, lo mejor
de todo es que todos ellos son texto plano así que podemos abrirlos y
mirar qué contienen:

\begin{itemize}
\item
  \lstinline!aux!: sirve para gestionar las referencias cruzadas
  (\lstinline!\ref!) y las citas bibliográficas (\lstinline!\cite!),
  entre otras cosas. En general, guarda la información que ha de pasarse
  de un proceso de compilación a otro.
\item
  \lstinline!log!: aquí se guarda la información sobre el proceso de
  compilación, las advertencias, los errores, los paquetes utilizados
  con su respectiva versión y tal. Es especialmente útil si nos falla al
  crear el documento y no sabemos por qué.
\item
  \lstinline!synctex!: sincroniza nuestro código fuente con el documento
  de salida en el IDE, es decir, al movernos por el código fuente vemos
  al lado la parte correspondiente del \emph{pdf}. Podemos borrarlo
  alegremente, simplemente se creará otro.
\item
  \lstinline!toc!, \lstinline!lof!, \lstinline!lot! \ldots{} : sirven
  para crear el índice, la lista de figuras, la lista de tablas \ldots{}
\item
  \lstinline!out!: lo genera \lstinline!hyperref! y vale para crear los
  enlaces que nos llevan de un lado a otro en el \emph{pdf}.
\item
  \lstinline!bbl!: es el archivo de bibliografía creado por BibTeX.
\item
  \emph{Otros}: otros paquetes crean archivos auxiliares para gestionar
  sus cosas, por ejemplo, el paquete
  \href{http://ctan.org/pkg/listofsymbols}{\lstinline!listofsymbols!}
  crea un \lstinline!sub! y un \lstinline!sym! para gestionar la lista
  de subíndices y de símbolos respectivamente. Otro ejemplo son los
  archivos \lstinline!idx! que sirven para hacer el índice por palabras
  con
  \href{https://en.wikibooks.org/wiki/LaTeX/Indexing}{\lstinline!makeindex!}.
\end{itemize}

Para el caso de Beamer se crean además estos otros:

\begin{itemize}
\item
  \lstinline!snm!: ayuda en el proceso de insertar imágenes en el tipo
  \lstinline!article!
\item
  \lstinline!nav!: sirve para crear la barra de navegación en la
  presentación
\end{itemize}

Todos estos archivos aparecen porque cuando nosotros le damos al
botoncillo para que nos cree el documento por dentro en realidad se
invocan un montón de procesos. Tomando como ejemplo un caso en el que
usemos \lstinline!pdflatex! y tengamos referencias se haría algo así:

\begin{lstlisting}[language=bash]
pdflatex fuente.tex  # primera compilación
bibtex fuente.aux    # la bibliografía
pdflatex fuente.tex  # para referencias 
pdflatex fuente.tex  # para referencias 
\end{lstlisting}

Como veis, necesitamos compilar tres veces para tener las referencias
correctamente. A este proceso se le pueden añadir otros pasos como
\lstinline!makeindex!, para el índice por palabras, o
\lstinline!makeglossaries!, para crear el glosario. Es por esta ristra
de procesos que nosotros no vemos que se escriben todos los archivos
auxiliares.

Os preguntaréis: \emph{¿estas mierdas para qué las va dejando por ahí?
¿no las puede crear y luego borrarlas?} Podría hacerlo, sí, Pandoc lo
hace de hecho, crea una carpeta temporal con las basuras y luego la
borra. Esto es como todo en la vida, tiene sus ventajas y sus
inconvenientes. Tener esos archivos auxiliares nos permite hurgar y
trucar cosas y es útil para ver dónde falla al compilar. Pero si no os
gusta que os deje cosas por ahí tenéis varias opciones:

\begin{itemize}
\item
  Si usáis un IDE tipo Kile, tendréis un botón para eliminar los
  archivos auxiliares.
\item
  Si el documento no tiene referencias cruzadas, ni citas
  bibliográficas, ni ninguna cosa que necesite archivos auxiliares
  podemos usar \lstinline!\nofiles! en el preámbulo y solo nos creará el
  \lstinline!pdf!y el \lstinline!log!. También se creará el
  \lstinline!synctex! si compilamos desde un IDE. Tampoco nos ahorramos
  mucha cosas como veis.
\item
  Añadir \lstinline!--aux-directory=CARPETA! a la orden de compilar. Así
  meterá todos los archivos auxiliares en la carpeta que le hemos dicho
  y buscará también ahí cuando los necesite. Lo podemos cambiar
  fácilmente en el IDE donde aparecen las órdenes. Esta es la opción que
  yo suelo usar (ahora que la he descubierto).
\item
  Compilar con la opción \lstinline!--jobname=CARPETA/OUTPUT INPUT.tex!,
  así guardará el \emph{pdf} y los archivos auxiliares en la carpeta que
  nosotros queramos\footnote{¡Gracias Shevek por hablarme de esta
    opción!}.
\item
  Usar \emph{Pandoc}. Esto nos hace cambiar nuestra forma de trabajar
  radicalmente, pero no nos quedan sobras por ahí.
\end{itemize}

\section{Referencias}

\href{http://www.dickimaw-books.com/latex/novices/html/auxiliary.html}{\emph{Auxiliary
Files} en Dickimaw Books}

\href{http://tex.stackexchange.com/questions/11123/prevent-pdflatex-from-writing-a-bunch-of-files}{\emph{Prevent
pdflatex from writing a bunch of files} en TeXExchange}

\href{http://stackoverflow.com/questions/3745908/i-dont-want-the-aux-log-and-synctex-gz-files-when-using-pdflatex}{\emph{I
don't want the .aux, .log and .synctex.gz files when using pdflatex} en
StackOverflow}

\href{http://latex-mk.sourceforge.net/}{\lstinline!latex-mk!}
